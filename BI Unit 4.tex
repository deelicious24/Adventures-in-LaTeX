\documentclass[12pt,a4paper]{article}
\begin{document}
\title{\textbf {Bahasa Indonesia}}
\date{Unit 4}
\maketitle
\tableofcontents{}
\section*{Notes}
\textit{This study guide was prepared by Dyandra Charismiadji based on information provided in "Bahasa dan Sastra Indonesia 2". It is provided as-is and may contain errors, though effort has been made to minimize this. }

\newpage
\section{Menulis Notula Rapat}
Kerangka umum sebuah notula adalah sebagai berikut:
\begin{itemize}
    \item \textbf {Rapat ke}:
    \item \textbf {Sifat rapat}: 
    \item \textbf {Hari}:
    \item \textbf {Tanggal}:
    \item \textbf {Pukul}:
    \item \textbf {Tempat}:
    \item \textbf {Jumlah yang diundang}:
    \item \textbf {Jumlah yang hadir}
    \item \textbf {Jabaran acara}:
    \item \textbf {Kesimpulan/keputusan}:
    \item \textbf {Rapat ditutup pukul}:
    \item \textbf {Pembuat notula},
\end{itemize}

\newpage
\section{Kalimat Ambigu}
Kalimat ambigu merupakan kalimat yang mengandung makna ganda yang dapat membingungkan pembaca. Penyebab kalimat ambigu dapat berupa:
\begin{itemize}
    \item \textbf {Penempatan keterangan subjek yang terlalu jauh}
    \item \textbf {Tidak digunakannya tanda baca koma}
    \item \textbf {Tidak digunakannya tanda hubung}
    \item \textbf {Penggunaan imbuhan atau kata yang memiliki rujukan ganda}
\end{itemize}
Contoh kalimat ambigu adalah:
\begin{itemize}
    \item Pembela yang mendampingi terdakwa Reva binti Rambe dalam perkara pencurian harta Ny. Najlawati, Selina Sumanto, S.H. dan Raffa Suherman, S.H., mengajukan protes ke Polda Metro Jaya
    \item Menurut adik Bintang Yustinus Woenarta sakit keras.
    \item Istri insinyur perangkat lunak yang menawan itu sedang minum bir.
    \item Nunu menelpon Louis yang sedang berada di rumah ibunya.
\end{itemize}

\newpage
\section{Mengidentifikasi Alur, Penokohan, dan Latar, dalam Cerpen yang Dibacakan}
\subsection{Watak dan Penokohan}
\begin{itemize}
    \item \textbf{Sifat tokoh}: Gambaran jalan pikiran dan reaksi tokoh terhadap perilaku yang terjadi
    \item \textbf{Peran tokoh}: Protagonis, antagonis, tritagonis
\end{itemize}

\subsection{Alur}
Ada tiga macam alur dalam sebuah cerita:
\begin{itemize}
    \item \textbf{Alur maju}
    \item \textbf{Alur mundur}
    \item \textbf{Alur gabungan}
\end{itemize}
Alur cerita dapat dibagi menjadi beberapa tahapan:
\begin{itemize}
    \item \textbf{Pengantar}
    \item \textbf{Penampilan masalah}
    \item \textbf{Puncak masalah}
    \item \textbf{Ketegangan menurun}
    \item \textbf{Penyelesaian}
\end{itemize}

\subsection{Latar}
Latar dapat dibagi menjadi tiga:
\begin{itemize}
    \item \textbf {Latar tempat}
    \item \textbf {Latar waktu}
    \item \textbf {Latar suasana}
\end{itemize}
\end{document}